\documentclass[11pt]{article}
\usepackage{amsfonts}
\usepackage{amsmath}
\usepackage{enumerate}
\usepackage{listings}

\topmargin -.5in
\textheight 9in
\oddsidemargin -.25in
\evensidemargin -.25in
\textwidth 7in

\begin{document}
% ===========QUESTION 1============
\begin{center}
  \Large\textbf{CS641: Level 1}\\
  \Large\textbf{Team: TrojanHorse}\\
  \medskip
  \large\textit{Members:\quad}\\
  \large\textit{Ayush Kumar, 180174\quad}\\
  \large\textit{Rishav Kumar, 180612\quad}\\
  \large\textit{Nilay Majorwar, 180483\quad}\\
  \medskip
  \large\textit{February 29, 2020\quad}
\end{center}

\bigskip
\bigskip

\textbf{Reaching the instructions: }
\medskip

Reaching the instructions was tricky in this level:
\begin{enumerate}
  \item Speak \texttt{go} at the first screen.
  \item \texttt{dive} into the lake.
  \item Diving more, we run out of breath and die. So instead, go \texttt{back} to the surface.
  \item \texttt{dive} back into the lake, and \texttt{pick} the wand.
  \item Go \texttt{back} to the surface, and go back to level 3.
  \item \texttt{wave} the wand in front of the two holes. The spirit is freed, and follows us now.
  \item Go to level 4, and \texttt{read} the glass panel to get the instructions. 
\end{enumerate}

\bigskip
\bigskip

\textbf{Instructions on glass panel: }

\begin{center}
  \texttt{
    "This is a magical screen. You can whisper something close to the screen and the corresponding coded text would appear on it after a while. So go ahead and try to break the code! The code used for this is a 4-round DES, so it should be easy for you!! Er wait ... maybe it is a 6-round DES ... sorry, my memory has blurred after so many years. But I am sure you can break even 6-round DES easily. A 10-round DES is a different matter, but this one surely is not 10-round ...(long pause) ... at least that is what I remember. One thing that I surely remember is that you can see the coded password by whispering 'password'. There was something funny about how the text appears, two letters for one byte or something like that. I do not recall more than that. I am sure you can figure it out though ..."
  }
\end{center}

\bigskip
\bigskip

\textbf{Breaking the alphabet-binary encoding: }
\medskip

It was told in the classroom that it's a 3-round DES. We observed that on entering a plaintext of length 1-16 i.e. we get ciphertext of length 16. Thus, 16 characters has to correspond to one full block. Since one full block is 8-byte long, 2 characters have to correspond to one byte. 

Also notice that the output contains of only 16 letters - from \texttt{f} to \texttt{u}. Now, 16 letters can be represented with 4 bits, so by concatenating the 4-bit formats of two characters, we get the corresponding byte.

In short, 

\begin{center}
  \texttt{plaintext[i]= (inputtext[2*i]-'f')*16 + (inputtext[2*i+1]-'f')}
\end{center}
\medskip
\pagebreak

\textbf{Getting the round three key: }
\medskip

We will launch a differential cryptanalysis attack to extract the key of the third round. We generated 10 differential pairs (see \texttt{third.py}), and applied the inverse of the initial permutation \texttt{ip} to get the texts we need to enter into the game (see \texttt{cio.py}). We note the obtained ciphertext pairs to get the full differential text pairs. Now we can launch the attack.
\medskip

Now consider the case of a particular differential pair. As taught in class, we find the xor of outputs of permutation box of round 3 (using xor of $L_2$). From this, we find the output xor of S-boxes. Also, we know the individual outputs of the expansion box of round 3, and can obtain the xor of the inputs of the S-boxes.
\medskip

So we know the input xor and output xor of each S-box. We will get a maximum of 64 input pair possibilities for each S-box, and thus 64 possibilities for that block of the key. Doing this for each differential pair, the intersection of all these possibility sets reduces to one value. Concatenating these key blocks, we get the full key for the third round. (see \texttt{third.py})
\medskip
\bigskip

\textbf{Getting the first two round keys: }
\medskip

Once we get the third round key, the problem reduces to a 2-round DES. We generate 6-7 plaintexts with corresponding ciphertexts (see \texttt{breakdes.py}). For a 2-round DES, we know the text of the middle stage too (due to the L-R swap). So we have the general problem of solving a 1-round DES with known plaintexts and ciphertexts.
\medskip

We can easily get the output of S-boxes, and will get 4 possibilities for the input of each S-box, which is the same as the output of key xor. We also know the input of key xor (which is just the expansion of $R_0$). So, we get 4 possibilities for the corresponding block of the key. Doing this for each plaintext-ciphertext, we reduce each set of possibilities to one, and concatenating these key blocks, we get the round key (see \texttt{breakdes.py}).
\medskip

Using this algorithm, we get the round keys for both the first round and second round. Thus, we obtain all the keys for the three-round DES.
\medskip

\begin{center}
  \texttt{KEY 1: 110011101111001010111101011010011111000000101000}

\texttt{KEY 2: 111111011100010111100111000110101100111000101001}

\texttt{KEY 3: 111100111100101110011011000110100101110100010000}
\end{center}
\medskip

\bigskip
\textbf{Getting the decrypted password:}
\medskip

Now, we just need to decrypt the encrypted password (\texttt{jfkifgqoorgrrnqnmssiiqislkftkurl}). Since the encrypted password has 32 characters, the decrypted password consists of two blocks. Splitting the encrypted password into blocks, we decrypt each block to get \texttt{rqmjnkisnpqirgpl} and \texttt{prrjfosfspqomhik}. Concatenating the two blocks, we get \texttt{rqmjnkisnpqirgplprrjfosfspqomhik}, which on entering in the game, takes us to the next level.

$$--------------------$$

\end{document}
\grid
\grid